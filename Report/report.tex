\documentclass{article}

% if you need to pass options to natbib, use, e.g.:
%     \PassOptionsToPackage{numbers, compress}{natbib}
% before loading neurips_2020

% ready for submission
% \usepackage{neurips_2020}

% to compile a preprint version, e.g., for submission to arXiv, add add the
% [preprint] option:
%     \usepackage[preprint]{neurips_2020}

% to compile a camera-ready version, add the [final] option, e.g.:
%     \usepackage[final]{neurips_2020}

% to avoid loading the natbib package, add option nonatbib:
     \usepackage[final, nonatbib]{neurips_2020}

\usepackage[utf8]{inputenc} % allow utf-8 input
\usepackage[T1]{fontenc}    % use 8-bit T1 fonts
\usepackage{hyperref}       % hyperlinks
\usepackage{url}            % simple URL typesetting
\usepackage{booktabs}       % professional-quality tables
\usepackage{amsfonts}       % blackboard math symbols
\usepackage{nicefrac}       % compact symbols for 1/2, etc.
\usepackage{microtype}      % microtypography

\title{ECSE 551 Mini Project 1}

% The \author macro works with any number of authors. There are two commands
% used to separate the names and addresses of multiple authors: \And and \AND.
%
% Using \And between authors leaves it to LaTeX to determine where to break the
% lines. Using \AND forces a line break at that point. So, if LaTeX puts 3 of 4
% authors names on the first line, and the last on the second line, try using
% \AND instead of \And before the third author name.

\author{
  Isabel Lougheed,~260989364 \\
  \And
  Mathieu Mailhot,~ \\
  \And
  Frank-Lucas Pantazis,~ \\
  % examples of more authors
  % \And
  % Coauthor \\
  % Affiliation \\
  % Address \\
  % \texttt{email} \\
  % \AND
  % Coauthor \\
  % Affiliation \\
  % Address \\
  % \texttt{email} \\
  % \And
  % Coauthor \\
  % Affiliation \\
  % Address \\
  % \texttt{email} \\
  % \And
  % Coauthor \\
  % Affiliation \\
  % Address \\
  % \texttt{email} \\
}

\makeatletter
\renewcommand{\@noticestring}{}  % Removes footer text
\makeatother

\begin{document}

\maketitle

\begin{abstract}
  This project involves implementing a logistic regression linear classifier from scratch.  This report presents the results of the logistic regression classifier model when performed on two datasets, a Chronic Kidney Disease (CKD) dataset and a battery dataset.  INCLUDE IMPORTANT FINDINGS.
\end{abstract}

\section{Introduction}

\section{Datasets}


The first dataset used is a CKD dataset that is comprised of 28 numerical features which each represent a medical measurement of a patient.  There is one target variable indicating whether the patient was diagnosed with CKD ('CKD') or not diagnosed with CKD ('Normal').  WRITE MORE ABOUT OUR FINDINGS ABOUT THE DATASET.
\\
\\
The second dataset used in this project is a battery dataset comprised of 32 real-valued features which represent specific battery attributes.  There is a target variable to classify whether the battery is normal ('Normal') or defective ('Defective').  WRITE MORE ABOUT OUR FINDINGS ABOUT THE DATASET.


\section{Results}

\section{Discussion and Conclusion}

\section{Statement of Contributions}

\section{Appendix}

\section*{References}

References follow the acknowledgments. Use unnumbered first-level heading for
the references. Any choice of citation style is acceptable as long as you are
consistent. It is permissible to reduce the font size to \verb+small+ (9 point)
when listing the references.
{\bf Note that the Reference section does not count towards the eight pages of content that are allowed.}
\medskip

\small

[1] Alexander, J.A.\ \& Mozer, M.C.\ (1995) Template-based algorithms for
connectionist rule extraction. In G.\ Tesauro, D.S.\ Touretzky and T.K.\ Leen
(eds.), {\it Advances in Neural Information Processing Systems 7},
pp.\ 609--616. Cambridge, MA: MIT Press.

[2] Bower, J.M.\ \& Beeman, D.\ (1995) {\it The Book of GENESIS: Exploring
  Realistic Neural Models with the GEneral NEural SImulation System.}  New York:
TELOS/Springer--Verlag.

[3] Hasselmo, M.E., Schnell, E.\ \& Barkai, E.\ (1995) Dynamics of learning and
recall at excitatory recurrent synapses and cholinergic modulation in rat
hippocampal region CA3. {\it Journal of Neuroscience} {\bf 15}(7):5249-5262.

\end{document}